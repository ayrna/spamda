% !TeX spellcheck = en_GB
%\documentclass[a4paper,10pt]{article}
\documentclass[a4paper,twoside,11pt]{article}

\usepackage[utf8x]{inputenc}
%\usepackage[latin9]{inputenc}
\usepackage[T1]{fontenc}
\usepackage[english]{babel}
\usepackage{gensymb}
\usepackage{palatino}
\usepackage{verbatim}
\usepackage{color}
\usepackage{upgreek}
\usepackage{amsmath}
\usepackage{dsfont}
\usepackage{eurosym}
\usepackage{url}
% \usepackage[tight]{subfigure}
\usepackage{fancyhdr}
\usepackage{fancybox}
\usepackage{fancyvrb}
\usepackage{a4wide}
\usepackage[authoryear]{natbib}
\usepackage[misc,geometry]{ifsym} 
%\usepackage{wrapfig}

\usepackage{graphicx}
\usepackage[caption=false]{subfig}

% Text color
\usepackage{color}
% For colors in Tables
\usepackage{colortbl} 
% Multirow
\usepackage{multirow}
% Long Tables
\usepackage{longtable} 
% Dash hline in tables
\usepackage{arydshln}
\usepackage{tcolorbox}
\usepackage[parfill]{parskip}
\usepackage{lineno}
\usepackage{bm}
\usepackage[mathscr]{euscript}

%\usepackage{graphicx}
%\usepackage[caption=false]{subfig}

\usepackage[linktoc=all,bookmarks,bookmarksopen=true,bookmarksnumbered=true]{hyperref}
%\usepackage[linktoc=page,bookmarks,bookmarksopen=true,bookmarksnumbered=true]{hyperref}

%\setcounter{secnumdepth}{4}
%\setlength{\topmargin}{0cm}
%\setlength{\textheight}{23cm}
%\setlength{\textwidth}{16cm}
%\setlength{\oddsidemargin}{0.5cm}
%\setlength{\evensidemargin}{-0.4cm}
%\setlength{\marginparsep}{0cm}
%\setlength{\marginparwidth}{0cm}
%\setlength{\parskip}{4mm}
%\setlength{\baselineskip}{3cm}
%\setlength{\headsep}{1.3cm}
%\setlength{\unitlength}{1cm}
%%\setlength{\footskip}{1 cm}
%%\setlength{\headheight}{0cm}
%%\setlength{\headsep}{0cm}
%\setlength{\tolerance}{1000}
%\setlength{\pretolerance}{2000}
%\pagestyle{headings}

%\newcommand\myeq{\stackrel{\text{\tiny def}}{=}}

\newcommand{\myAuthors}{Antonio Manuel Gómez-Orellana$^1$, Juan Carlos Fernández$^1$, Manuel Dorado-Moreno$^1$, Pedro Antonio Gutiérrez$^1$ and César Hervás-Martínez$^1$.}

\newcommand{\myAuthorsShort}{A. M. Gómez-Orellana$^1$, J. C. Fernández$^1$, M. Dorado-Moreno$^1$, P. A. Gutiérrez$^1$ and C. Hervás-Martínez$^1$.}

\newcommand{\myEmail}{\Letter \quad  am.gomez@uco.es, jfcaballero@uco.es, manuel.dorado@uco.es, pagutierrez@uco.es, chervas@uco.es}

\newcommand{\myTitle}{Building Suitable Datasets for Soft Computing and Machine Learning Techniques from Meteorological Data Integration: A Case Study for Predicting Significant Wave Height and Energy Flux}

\newcommand{\myShortTitle}{Response to reviewers}

\newcommand{\myJournal}{\textbf{Energies}}

\newcommand{\myRefPaper}{\textbf{Paper reference: energies-1069865}}

\newcommand{\myDepartament}{$^1$Department of Computer Science and Numerical Analysis, University of Cordoba, 14071, Cordoba, Spain.}

\definecolor{gray090}{gray}{0.90}


%%%%%%%%%%%% headings setup %%%%%%%%%%%%
\pagestyle{fancy}
% clear styles
\fancyhf{}
% odd pages -> chapter name
\fancyhead[LO]{\leftmark}
% even pages
%\fancyhead[RE]{\myJournal -- \myShortTitle}
\fancyhead[RE]{\myJournal}
% odd/even pages
\fancyhead[RO,LE]{\thepage}

\renewcommand{\sectionmark}[1]{\markboth{\textbf{\thesection. #1}}{}}
%\renewcommand{\subsectionmark}[1]{\markright{\textbf{\thesection. #1}}}

% line under headings
\renewcommand{\headrulewidth}{0.6pt}
% line over foot
%\renewcommand{\footrulewidth}{0.6pt}

% increases heading weight
\setlength{\headheight}{1.5\headheight}
\fancyfoot{}

\newcommand{\newtext}[1]{\textcolor{red}{#1}}

% New revisor comment command
\newcounter{ecomments}[section]

\newcommand{\ecomment}[1]
{
	\stepcounter{comments}
%	\addcontentsline{toc}{subsection}{Editor \arabic{section} Comment \arabic{comments}}
	\begin{tcolorbox}[colback=black!5,colframe=white!45!black,title=Comment \arabic{comments}]
		#1
	\end{tcolorbox}
}

% New revisor comment command
\newcounter{comments}[section]

\newcommand{\rcomment}[1]
{
	\stepcounter{comments}
%	\addcontentsline{toc}{subsection}{Reviewer \arabic{section} Comment \arabic{comments}}
	\vspace{0.6cm}
	\begin{tcolorbox}[colback=black!5,colframe=white!45!black,title=Comment \arabic{comments}]
		#1
	\end{tcolorbox}
}


\begin{document}

\thispagestyle{plain}

\begin{center}
	{\LARGE\myTitle} \vspace{0.5cm} \\
	{\LARGE\myJournal} \vspace{0.5cm} \\
	{\Large\myRefPaper} \vspace{0.5cm} \\
% 	\today \vspace{0.5cm} \\
	\myAuthors \vspace{0.5cm} \\
	{\myDepartament} \vspace{0.5cm} \\
	{\myEmail}
\end{center}

\tableofcontents

\section{Responses to the Editor}
We would like to thank the Editor and the anonymous reviewers for this new round of review. We have addressed the final points required by the reviewers. 

%\textcolor{red}{Summary of changes}

Please find below the responses to the reviewers' comments.

\section{Responses to reviewers' comments}
We sincerely thank the anonymous reviewers for their careful reviews, constructive comments and hints to improve the impact of this work. 

Most important changes highlighted in \textcolor{blue}{blue color} in the previous version of the manuscript have already been set to the default color (\textbf{black color}). Lines highlighted in \textcolor{blue}{blue color} correspond to the final points of this new revision. Please, find below detailed responses to the comments.

% Additionally, in order to improve the English quality and style we have carried out a proof-reading.
% Additionally, in order to improve the paper's readability, we have carried out a proof-reading, where different typos and weird expressions have been corrected.
%Additionally, in order to improve the english language and style we have carried out a review.

% The most important changes made in the first revision of the paper concerning the initial version are  \textcolor{blue}{marked in blue}, while the changes made in this second revision  \textcolor{red}{are marked in red}. Also, this letter response attempts to clarify the questions asked by the reviewers.

%\newpage
\section{Reviewer \#1}
\addtocounter{section}{-2}

%%%%%%%%%%%%%%%%%%%%%%%%%% Comment %%%%%%%%%%%%%%%%%%%%%%%%%%
\rcomment
{
Dear authors,
Thank you for your efforts in revising the manuscript. I believe you did a great job in the revision and the changes alleviated my concerns regarding the manuscript. Therefore, I recommend its publication.

\vspace{0.5cm}
Looking forward to seeing the published version of your paper.

\vspace{0.5cm}
Good luck
}

\textbf{Response:}
{
Thank you very much for the excellent review carried out to our work.
}
%%%%%%%%%%%%%%%%%%%%%%%%%% Comment %%%%%%%%%%%%%%%%%%%%%%%%%%







%%%%%%%%%%%%%%%%%%%% Another Reviewer %%%%%%%%%%%%%%%%%%%%
%\newpage
\clearpage
\addtocounter{section}{+2}
\section{Reviewer \#2}
\addtocounter{section}{-2}

%%%%%%%%%%%%%%%%%%%%%%%%%% Comment %%%%%%%%%%%%%%%%%%%%%%%%%%
\rcomment{
The manuscript is in a good shape. The manuscript is suitable for publication in its current form.
}

\textbf{Response:}
{
We would like to thank you again for the excellent review carried out to our work.
}
%%%%%%%%%%%%%%%%%%%%%%%%%% Comment %%%%%%%%%%%%%%%%%%%%%%%%%%



%%%%%%%%%%%%%%%%%%%% Another Reviewer %%%%%%%%%%%%%%%%%%%%
%\newpage
\clearpage

\addtocounter{section}{+2}
\section{Reviewer \#3}
\addtocounter{section}{-2}

%%%%%%%%%%%%%%%%%%%%%%%%%% Comment %%%%%%%%%%%%%%%%%%%%%%%%%%
\rcomment{
The manuscript is related to an interesting topic and based on open database.

\vspace{0.5cm}
The structure is organized in a good way. Few adjustments are requested to introduction and conclusions.

\vspace{0.5cm}
First of all, in the abstract openess of data is mentioned but it is not discussed deeply in stating the background. See recent articles published in top-ranked Journals such as Energy, Energies and others. As example, consider

\vspace{0.5cm}
https://doi.org/10.1016/j.energy.2020.118803 

\vspace{0.5cm}
https://doi.org/10.3390/en13226095

\vspace{0.5cm}
The research question formulated by the authors to fill the identified research gap should open the conclusions section as
well as reporting limitation and replicability potential of the answer.

\vspace{0.5cm}
Please check carefully English.
}

\textbf{Response:}
{
First of all, we would like to thank Reviewer \#3 for his/her suggestions.

We have mentioned the two sources of data from which the data integration using SPAMDA is carried out in \textbf{Section ''1. Introduction`` (lines 117-122 - ''The meteorological data ... build datasets``)}, but we have not made an extensive explanation of these data because they are well known. Data provided by the NDBC and the NNRP are available for the scientific community and there are manuals and detailed descriptions on the websites of both information sources. Even so, we have dedicated \textbf{Section ``2. Meteorological data sources''} to briefly discuss both sources of data and how they are used in SPAMDA, supporting this section with bibliographical references and website addresses. 
%Figure 1 and Table 1 are also included as additional explanations of both data sources.

We sincerely believe that the paper [11] indicated by the reviewer underlines the importance of having well-formatted and high-quality data to be able to use them with modelling techniques, as well as the importance of both data and modelling tools being available to the scientific community and decision-makers. This reinforces our work, so we have introduced this reference in \textbf{Section ''1. Introduction`` (lines 37-40 - ''In this sense ... decision-makers
[11]``)}. The reference [25] also provides background to our work, since it reflects the importance of the correct arrangement of WECs for a better obtaining of power output and energy. This reference has been introduced in \textbf{Section ''1. Introduction`` (line 69)}.

[11] Manfren, M.; Nastasi, B.; Groppi, D.; Astiaso Garcia, D. Open data and energy analytics - An analysis of essential information for energy system planning, design and operation. Energies 2020, 13(9), 2334.

[25] Amini, E.; Golbaz, D.; Amini, F.; Majidi Nezhad, M.; Neshat, M.; Astiaso Garcia, D. A Parametric Study of Wave Energy Converter Layouts in Real Wave Models. Energies 2020, 13(22), 6095.

Finally, we have modified Section ''5. Conclusions`` to begin with the research gap. The replicability of data used in Section ''4. A case study applied to Gulf of Alaska`` or any other dataset created with SPAMDA, can be obtained by providing the parameters entered in the tool or simply by providing the final dataset after the last data integration step. These clarifications have been included in \textbf{Section ''5. Conclusions`` (lines 755-763 - ''Studies on marine energy ... among others``, lines 780-783 - ''The final datasets ... SC or ML tool used``)}.

%As a conclusion, we believe that open data, open science, open innovation and, more specifically, open energy modelling principles are crucial to create an effective science-policy-market interaction in energy and sustainability transitions. Research efforts in energy modelling should be oriented to the identification of solutions that represent good compromises between connectivity and diversity and to the definition of appropriate hierarchies of data and models, for example by using linked open data schema and meta-models.


We would like to thank you again for the suggestions and review carried out to our work.


%%%%%%%%%%%%%%%%%%%%%%%%%% Comment %%%%%%%%%%%%%%%%%%%%%%%%%%

\end{document}
