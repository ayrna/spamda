% !TeX spellcheck = en_GB
%\documentclass[a4paper,10pt]{article}
\documentclass[a4paper,twoside,11pt]{article}

\usepackage[utf8x]{inputenc}
%\usepackage[latin9]{inputenc}
\usepackage[T1]{fontenc}
\usepackage[english]{babel}

\usepackage{palatino}
\usepackage{verbatim}
\usepackage{color}
\usepackage{upgreek}
\usepackage{amsmath}
\usepackage{dsfont}
\usepackage{eurosym}
\usepackage{url}
% \usepackage[tight]{subfigure}
\usepackage{fancyhdr}
\usepackage{fancybox}
\usepackage{fancyvrb}
\usepackage{a4wide}
\usepackage[authoryear]{natbib}
\usepackage[misc,geometry]{ifsym} 
%\usepackage{wrapfig}
%\usepackage{graphicx}

\usepackage{graphicx}
\usepackage[caption=false]{subfig}

% Text color
\usepackage{color}
% For colors in Tables
\usepackage{colortbl} 
% Multirow
\usepackage{multirow}
% Long Tables
\usepackage{longtable} 
% Dash hline in tables
\usepackage{arydshln}
\usepackage{tcolorbox}
\usepackage[parfill]{parskip}
\usepackage{lineno}
\usepackage{bm}
\usepackage[mathscr]{euscript}

%\usepackage{graphicx}
%\usepackage[caption=false]{subfig}

\usepackage[linktoc=all,bookmarks,bookmarksopen=true,bookmarksnumbered=true]{hyperref}

%\setcounter{secnumdepth}{4}
%\setlength{\topmargin}{0cm}
%\setlength{\textheight}{23cm}
%\setlength{\textwidth}{16cm}
%\setlength{\oddsidemargin}{0.5cm}
%\setlength{\evensidemargin}{-0.4cm}
%\setlength{\marginparsep}{0cm}
%\setlength{\marginparwidth}{0cm}
%\setlength{\parskip}{4mm}
%\setlength{\baselineskip}{3cm}
%\setlength{\headsep}{1.3cm}
%\setlength{\unitlength}{1cm}
%%\setlength{\footskip}{1 cm}
%%\setlength{\headheight}{0cm}
%%\setlength{\headsep}{0cm}
%\setlength{\tolerance}{1000}
%\setlength{\pretolerance}{2000}
%\pagestyle{headings}

%\newcommand\myeq{\stackrel{\text{\tiny def}}{=}}

\newcommand{\myAuthors}{Antonio Manuel Gómez-Orellana$^1$, Juan Carlos Fernández$^1$, Manuel Dorado-Moreno$^1$, Pedro Antonio Gutiérrez$^1$ and César Hervás-Martínez$^1$.}

\newcommand{\myAuthorsShort}{A. M. Gómez-Orellana$^1$, J. C. Fernández$^1$, M. Dorado-Moreno$^1$, P. A. Gutiérrez$^1$ and C. Hervás-Martínez$^1$.}

\newcommand{\myEmail}{\Letter \quad  am.gomez@uco.es, jfcaballero@uco.es, manuel.dorado@uco.es, pagutierrez@uco.es, chervas@uco.es}

\newcommand{\myTitle}{Building Suitable Datasets for Soft Computing and Machine Learning Techniques from Meteorological Data Integration: A Case Study for Predicting Significant Wave Height}

\newcommand{\myShortTitle}{Response to reviewers}

\newcommand{\myJournal}{Energies, Special Issue on ``Soft Computing Techniques in Energy System''}

\newcommand{\myRefPaper}{\textbf{Paper reference: energies-1023087}}

\newcommand{\myDepartament}{$^1$Department of Computer Science and Numerical Analysis, University of Cordoba, 14071, Cordoba, Spain.}

\definecolor{gray090}{gray}{0.90}


%%%%%%%%%%%% headings setup %%%%%%%%%%%%
\pagestyle{fancy}
% clear styles
\fancyhf{}
% odd pages -> chapter name
\fancyhead[LO]{\leftmark}
% even pages
%\fancyhead[RE]{\myJournal -- \myShortTitle}
\fancyhead[RE]{\myJournal}
% odd/even pages
\fancyhead[RO,LE]{\thepage}

\renewcommand{\sectionmark}[1]{\markboth{\textbf{\thesection. #1}}{}}
%\renewcommand{\subsectionmark}[1]{\markright{\textbf{\thesection. #1}}}

% line under headings
\renewcommand{\headrulewidth}{0.6pt}
% line over foot
%\renewcommand{\footrulewidth}{0.6pt}

% increases heading weight
\setlength{\headheight}{1.5\headheight}
\fancyfoot{}

\newcommand{\newtext}[1]{\textcolor{red}{#1}}

% New revisor comment command
\newcounter{ecomments}[section]

\newcommand{\ecomment}[1]
{
	\stepcounter{comments}
%	\addcontentsline{toc}{subsection}{Editor \arabic{section} Comment \arabic{comments}}
	\begin{tcolorbox}[colback=black!5,colframe=white!45!black,title=Comment \arabic{comments}]
		#1
	\end{tcolorbox}
}

% New revisor comment command
\newcounter{comments}[section]

\newcommand{\rcomment}[1]
{
	\stepcounter{comments}
%	\addcontentsline{toc}{subsection}{Reviewer \arabic{section} Comment \arabic{comments}}
	\vspace{0.6cm}
	\begin{tcolorbox}[colback=black!5,colframe=white!45!black,title=Comment \arabic{comments}]
		#1
	\end{tcolorbox}
}


\begin{document}

\thispagestyle{plain}

\begin{center}
	{\LARGE\myTitle} \vspace{0.5cm} \\
	{\LARGE\myJournal} \vspace{0.5cm} \\
	{\Large\myRefPaper} \vspace{0.5cm} \\
% 	\today \vspace{0.5cm} \\
	\myAuthors \vspace{0.5cm} \\
	{\myDepartament} \vspace{0.5cm} \\
	{\myEmail}
\end{center}

%\tableofcontents

\section{Responses to the Editor}
We would like to thank the journal's Editor for giving us the opportunity to improve our paper with this review. Following the Editor's indications, we have prepared a revised version of the paper, where we have addressed the points required by the reviewers to obtain an improved version of the manuscript.

Please find below the responses to the reviewers' comments.



\section{Responses to reviewers' comments}
We sincerely thank the anonymous reviewers for their careful reviews, constructive comments and hints to improve the impact of this work. We honestly believe that the quality of the paper has increased after addressing the comments and issues raised by the reviewers. 

Most important changes in the paper have been highlighted in \textcolor{blue}{blue color} in the revised version of the manuscript. Please, find below detailed responses to the comments.

Additionally, in order to improve the english language and style we have carried out a proof-reading.

% The most important changes made in the first revision of the paper concerning the initial version are  \textcolor{blue}{marked in blue}, while the changes made in this second revision  \textcolor{red}{are marked in red}. Also, this letter response attempts to clarify the questions asked by the reviewers.

% Most important changes in the paper have been highlighted in \textcolor{red}{red} color in the revised version of the manuscript. Please, find below detailed responses to the comments.

% Additionally, in order to improve the paper's readability, we have carried out a proof-reading, where different typos and weird expressions have been corrected.


\newpage
\section{Reviewer \#1}
\addtocounter{section}{-2}


%%%%%%%%%%%%%%%%%%%%%%%%%% Comment %%%%%%%%%%%%%%%%%%%%%%%%%%
\rcomment
{
This work develops a framework to collect, integrate, and preprocess meteorological observation data from NDBC and NNRP.
Moreover, this framework uses machine learning techniques to do predictions based on these pre-processed datasets. This work
saves researchers from tedious and repetitive data collection and pre-processing work. Also, the use of machine learning in this framework is very useful given that NDBC and NNRP contain missing data in their observation datasets.

\vspace{0.5cm}
My only comment is that the wave prediction from the machine learning technique is not well validated.
}


\textbf{Response:}
{
Thank you very much for your review, providing very interesting comments and suggestions. 

Bla, bla, bla. \textbf{These clarifications are mainly described in the paper on page 1, lines 1-33, on page 2, lines 95-112, on page 3, lines 136-151 and 165-181, as well as in the ``Conclusions'' Section}.
}
%%%%%%%%%%%%%%%%%%%%%%%%%% Comment %%%%%%%%%%%%%%%%%%%%%%%%%%







%%%%%%%%%%%%%%%%%%%% Another Reviewer %%%%%%%%%%%%%%%%%%%%
\newpage
\clearpage
\addtocounter{section}{+2}
\section{Reviewer \#2}
\addtocounter{section}{-2}

%%%%%%%%%%%%%%%%%%%%%%%%%% Comment %%%%%%%%%%%%%%%%%%%%%%%%%%
\rcomment{
Dear authors,
Thank you for submitting your paper to the Energies. I think your paper is an informative paper that can be published after a major revision.

\vspace{0.5cm}
It is not clear what the research gap that the paper is addressing. What is the objective of this paper? Please clarify somewhere clearly all your contributions.
}

\textbf{Response:}
{
First of all, we would like to thank Reviewer \#2 for his/her suggestions to improve the paper.

In our understanding, we think that this work contributes several ideas to the field of Machine Learning (ML):

\begin{itemize}
	\item Bla, bla, bla,
	\item Bla, bla, bla,
	\item Bla, bla, bla,
\end{itemize}

Bla, bla, bla, \textbf{This clarification has been rewritten in page 2, lines 53-61.}
}
%%%%%%%%%%%%%%%%%%%%%%%%%% Comment %%%%%%%%%%%%%%%%%%%%%%%%%%

%%%%%%%%%%%%%%%%%%%%%%%%%% Comment %%%%%%%%%%%%%%%%%%%%%%%%%%
\rcomment
{
The literature review is not goal-oriented. The process should be as follows:

\vspace{0.5cm}
i) Critical evaluation of the literature; ii) identifying the gap based on this critical evaluation of the literature; iii) proposing your hypothesis to address the identified gap; iv) posing the appropriate and relevant research question based on your proposed hypothesis, and finally explaining your proposed method to answer this research question. Therefore, you will have a systematic way of conducting your research. Right now, the literature review section has no clear objective.
}

\textbf{Response:}
{
Following the instructions of the reviewer, bla, bla, bla. \textbf{These clarifications have been included in the paper on page 11, lines 646-660, in the new Figure 6, and in the ``Conclusions'' Section, lines 672-682.}
}
%%%%%%%%%%%%%%%%%%%%%%%%%% Comment %%%%%%%%%%%%%%%%%%%%%%%%%%

%%%%%%%%%%%%%%%%%%%%%%%%%% Comment %%%%%%%%%%%%%%%%%%%%%%%%%%
\rcomment
{
A thorough editorial check and English improvement are needed. Please kindly proofread the entire manuscript.

}

\textbf{Response:}
{
A complete revision of the English has been carried out in this review step.
}
%%%%%%%%%%%%%%%%%%%%%%%%%% Comment %%%%%%%%%%%%%%%%%%%%%%%%%%

%%%%%%%%%%%%%%%%%%%%%%%%%% Comment %%%%%%%%%%%%%%%%%%%%%%%%%%
\rcomment
{
The conclusion part is also needed to be revised; which questions are answered, what is the value/originality/contribution of the paper, how the proposed method answers the research questions that previous methods are not able to answer?
}

\textbf{Response:}
{
Bla, bla, bla, 

Ok, good point. There are two main reasons for choosing
}
%%%%%%%%%%%%%%%%%%%%%%%%%% Comment %%%%%%%%%%%%%%%%%%%%%%%%%%

%%%%%%%%%%%%%%%%%%%%%%%%%% Comment %%%%%%%%%%%%%%%%%%%%%%%%%%
\rcomment
{
It feels you need a king of aggregating results somewhere clearer.

}

\textbf{Response:}
{
Bla, bla, bla. The reviewer is right about this comment on input variables.....
}
%%%%%%%%%%%%%%%%%%%%%%%%%% Comment %%%%%%%%%%%%%%%%%%%%%%%%%%

%%%%%%%%%%%%%%%%%%%%%%%%%% Comment %%%%%%%%%%%%%%%%%%%%%%%%%%
\rcomment
{
Please provide more explanation for Figure 21. Also, there are too many figures in the paper. Please reconsider combining some of them together.

}
\textbf{Response:}
{
Bl, bla, bla. Following this suggestion by the reviewer, we have changed...
}
%%%%%%%%%%%%%%%%%%%%%%%%%% Comment %%%%%%%%%%%%%%%%%%%%%%%%%%

%%%%%%%%%%%%%%%%%%%%%%%%%% Comment %%%%%%%%%%%%%%%%%%%%%%%%%%
\rcomment
{
An adequate literature review and a clear gap identification have been tried to be conducted. However, authors have ignored some research that has been done in the area. I strongly recommend the authors to provide a more comprehensive literature review in the introduction section. The following papers are recommended:

\begin{itemize}
    \item Significant wave height forecasting via an extreme learning machine model integrated with improved complete ensemble empirical mode decomposition. Renewable and Sustainable Energy Reviews, 104, 281-295. 
    \item A wavelet-Particle swarm optimization-Extreme learning machine hybrid modeling for significant wave height prediction. Ocean Engineering, 213, 107777.
    \item Managing computational complexity using surrogate models: a critical review. Research in Engineering Design, 31(3), 275-298.
    \item Near real-time significant wave height forecasting with hybridized multiple linear regression algorithms. Renewable and Sustainable Energy Reviews, 132, 110003.
    \item Significant wave height and energy flux prediction for marine energy applications: A grouping genetic algorithm–Extreme Learning Machine approach. Renewable Energy, 97, 380-389.
    \item Outlook on biofuels in future studies: A systematic literature review. Renewable and Sustainable Energy Reviews, 134, 110326.
    \item Statistical models for improving significant wave height predictions in offshore operations. Ocean Engineering, 206, 107249.
    \item Regional ocean wave height prediction using sequential learning neural networks. Ocean Engineering, 129, 605-612.
    \item Ensemble of surrogates and cross-validation for rapid and accurate predictions using small data sets. AI EDAM, 33(4), 484-501.
    \item Prediction of significant wave height; comparison between nested grid numerical model, and machine learning models of artificial neural networks, extreme learning and support vector machines. Engineering Applications of Computational Fluid Mechanics, 14(1), 805-817.
\end{itemize}
}

\textbf{Response:}
{
Bla, bla, bla,
}
%%%%%%%%%%%%%%%%%%%%%%%%%% Comment %%%%%%%%%%%%%%%%%%%%%%%%%%

%%%%%%%%%%%%%%%%%%%%%%%%%% Comment %%%%%%%%%%%%%%%%%%%%%%%%%%
\rcomment
{
If you can, please make a small comparison between what did you do and what others did before, as a conclusion.

}

\textbf{Response:}
{
Bla, bla, bla,
}
%%%%%%%%%%%%%%%%%%%%%%%%%% Comment %%%%%%%%%%%%%%%%%%%%%%%%%%

%%%%%%%%%%%%%%%%%%%%%%%%%% Comment %%%%%%%%%%%%%%%%%%%%%%%%%%
\rcomment
{
The abstract is not deep enough and Is not well prepared. Please try to re-write it better. The problem should be clearly stated and the gap in which you are going to address the need to be clarified. Simply explain your contributions and key findings.
}

\textbf{Response:}
{
Bla, bla, bla. The reviewer is completely right. We have rewritten the abstract to better describe the application tackled.
}
%%%%%%%%%%%%%%%%%%%%%%%%%% Comment %%%%%%%%%%%%%%%%%%%%%%%%%%

%%%%%%%%%%%%%%%%%%%%%%%%%% Comment %%%%%%%%%%%%%%%%%%%%%%%%%%
\rcomment
{
There are some errors in your reference list. Please check and fix the errors.

}


\textbf{Response:}
{
Bla, bla, bla. Thanks for this comment. The distribution....

\vspace{0.5cm}
We would like to thank you again for the excellent review carried out to our work.
}
%%%%%%%%%%%%%%%%%%%%%%%%%% Comment %%%%%%%%%%%%%%%%%%%%%%%%%%



%%%%%%%%%%%%%%%%%%%% Another Reviewer %%%%%%%%%%%%%%%%%%%%
\newpage
\clearpage

\addtocounter{section}{+2}
\section{Reviewer \#3}
\addtocounter{section}{-2}

%%%%%%%%%%%%%%%%%%%%%%%%%% Comment %%%%%%%%%%%%%%%%%%%%%%%%%%
\rcomment{
The Authors present a new open source tool for the creation of datasets integrated by meteorological variables from two sources
of information. Basically, a user-friendly software has been developed and described on the basis of pivotal parameters, carefully selected for the specific study of the wave height in the Gulf of Alaska. The statistical model can be in principle extended to other meteorological measurements and monitoring. 

\vspace{0.5cm}
This study is very technical and of interest for a specific audience. By the way, I don’t see any link between the topic of the manuscript and the journal Energies.

\vspace{0.5cm}
The software developed it is clearly presented as useful tool for meteorological application and no application to energy saving, production, conversion or similar it is presented.

\vspace{0.5cm}
Thus, I recommend to submit the menuscript to a more specialized journal.
}

\textbf{Response:}
{
First of all, we would like to thank Reviewer \#3 for his/her suggestions to improve the submitted manuscript.

\vspace{0.5cm}
VAMOS A VER PEDAZO DE HIJO DE PUTA, TE HAS QUERIDO QUITAR EL LEER LA REVISION DE UN PLUMAZO
}
%%%%%%%%%%%%%%%%%%%%%%%%%% Comment %%%%%%%%%%%%%%%%%%%%%%%%%%

\end{document}
