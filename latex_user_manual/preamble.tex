
% Contenido del preámbulo del documento LaTeX

\usepackage[utf8]{inputenc}
\usepackage{color}
\usepackage{tabulary}
\usepackage[table]{xcolor}
\usepackage[T1]{fontenc}		% Para poner los símbolos << y >>
\usepackage{setspace}
\usepackage{pstricks}
\usepackage{fancyhdr}
\usepackage{tocbibind}			% Incluye bibliografía en índice
\usepackage{natbib}
\bibliographystyle{unsrt}		% Estilo de la bibliografía.
\usepackage{fancyvrb}			% Para usar Verbatim
\usepackage{bookmark}			% Para insertar la bibliografía al mismo nivel que el resto de capítulos.
\usepackage{textcomp} 			% Para usar \textquotesingle
\usepackage{pdfpages}
\usepackage{epstopdf}
\usepackage{tikz} 				% Para posicionar imágenes.
\usepackage{gensymb} 			% Para poner el símbolo de los grados.

\usepackage{changepage}			% Para usar {adjustwidth}
\usepackage{tocbibind}			% Incluye bibliografía en índice

\usepackage{indentfirst}		% Indenta el primer párrafo de cada sección.

\usepackage{lineno}
\usepackage{caption}
\usepackage{appendix}

\usepackage{tcolorbox}

% Macro para los mensajes de tipo Warning.
\newtcolorbox{warningbox}[2]
{
	title = \textbf{#1},
	text width=#2,
	colback=red!4!white,
	colframe=orange!65!white
}

% Color gris para las filas de la tabla.
\definecolor{gray090}{gray}{0.90}

% Carpeta en la que se crean los .pdf a partir de las imágenes .eps
\epstopdfsetup{outdir=./figures_converted_to_pdf/}

\DeclareGraphicsExtensions{.eps}


% Espaciado de párrafo.
\parskip=5pt


% Indentación a la izquierda en cada comienzo de párrafo.
\parindent=30pt


% Márgenes del documento.
\usepackage[left=3.5cm, right=2.5cm, top=2.5cm, bottom=3.5cm]{geometry}


% Configuración del formato para la cabecera de los títulos.
\makeatletter
\def\LigneVerticale{\vrule height 4cm depth 2cm\hspace{0.1cm}\relax}

\def\LignesVerticales{%
  \let\LV\LigneVerticale\LV\LV\LV\LV\LV\LV\LV\LV\LV\LV}
  
\def\GrosCarreAvecUnChiffre#1{%
  \rlap{\vrule height 0.8cm width 1cm depth 0.2cm}%
  \rlap{\hbox to 1cm{\hss\mbox{\white #1}\hss}}%
  \vrule height 0pt width 1cm depth 0pt}
  
\def\@makechapterhead#1{\hbox{%
    \huge
    \LignesVerticales
    \hspace{-0.5cm}%
    \GrosCarreAvecUnChiffre{\thechapter}
    \hspace{0.4cm}\hbox{#1}%
}\par\vskip 2cm}

%\def\@makeschapterhead#1{\hbox{%
%    \huge
%    \LignesVerticales
%    \hspace{0.5cm}
%    \hbox{#1}%
%}\par\vskip 2cm}



% Configuración del formato de cabecera y pie de página.
\pagestyle{fancy}
\fancyhf{}							% Elimina la configuración actual de la cabecera.
\fancyfoot{}						% Elimina la configuración actual del pie de página.
\fancyhead[LO]{\leftmark}	 	% Páginas impares ........: en la parte izquierda del encabezado aparecerá el nombre del capítulo.
\fancyhead[RE]{\rightmark}	 	% Páginas pares ..........: en la parte derecha del encabezado aparecerá el nombre de la sección.
\fancyfoot[RO,LE]{\thepage} 	% Númeración de páginas ..: impares en la derecha, pares en la izquierda.


% Para no incluir el encabezado en páginas en blanco.
\makeatletter
  \def\cleardoublepage{\clearpage\if@twoside \ifodd\c@page\else
  \vspace*{\fill}
    \thispagestyle{empty}
    \newpage
    \if@twocolumn\hbox{}\newpage\fi\fi\fi}
\makeatother


% Comando para insertar una página vacía completa.
\newcommand{\paginavaciacompleta}{\newpage{\thispagestyle{empty}\cleardoublepage}}


\hyphenation{matching}
