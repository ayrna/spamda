
\chapter{Antecedentes}

	\begin{onehalfspace}
	
		Se puede considerar como antecedente de referencia la herramienta \textit{Scientific Processing of Meteorological Data} (SPMD). Así mismo, se han buscado aplicaciones que generen conjuntos de datos creados a partir de la información obtenida del NDBC y NNRP, pero no se tiene conocimiento de su existencia. Sin embargo, sí se han encontrado algunas aplicaciones que trabajan con ficheros en formato NetCDF.
		
		\section{ncdump}

			La herramienta \textit{ncdump} \cite{Ncdump2017} permite generar una representación del contenido de un fichero NetCDF empleando \textit{Common Data form Language} (CDL), la cual puede ser visualizada por pantalla, editada o servir de entrada para generar un nuevo fichero.
			
			Dispone de diferentes opciones con las cuales es posible configurar la salida de dicha representación, así como seleccionar qué contenidos del fichero se desean procesar.
			
			Es muy versátil para analizar el contenido de este tipo de ficheros y almacenarlo en un formato de texto plano, consiguiendo de este modo reducir la complejidad a una dimensión.
			
			La ejecución de la misma se realiza desde la línea de comandos y no dispone de interfaz gráfica. En la figura \ref{figura:ncdump} se muestra el uso de \textit{ncdump} para obtener el contenido de la cabecera de un fichero NetCDF. En ella se puede observar que la dimensionalidad del fichero es de 1465 (número de mediciones) x 2 (latitud Norte: 57.5\degree \space y 55.0\degree) x 2 (longitud Este: 210\degree \space y 212.5\degree).
			
			
			\begin{figure}
				\centering
			   \includegraphics[scale=0.41]{img/antecedentes/ncdump.png}
				\caption{Uso de \textit{ncdump} para mostrar la cabecera de un fichero NetCDF}
				\label{figura:ncdump}
			\end{figure}

	\end{onehalfspace}
